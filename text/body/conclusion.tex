\section{Conclusion} \label{sec:conclusion}

In the context of evolutionary computation, the existing approach to collecting phylogenetic history is emblematic of this existing complete-observability paradigm.
Historically, most research using ABM has assumed complete observability of model state.
Indeed, the ability to measure properties \textit{in silico} that would be impossible to observe \textit{in vitro} or \textit{in vivo} is a major benefit of scientific work using ABM.
To advance on this front, we propose a fundamental re-frame of simulation that shifts from a paradigm of ``complete,'' deterministic observability of simulation state to instead collect data through dynamic, partial, and potentially best-effort sampling akin to approaches traditionally used to study real-world systems (e.g., ice core samples, paleontological fossils).
The aim of this strategy is to resolve scaling bottlenecks by economizing use of interconnect bandwidth, memory, and disk storage storage and better tolerating intermittent disruption.
Trading a controlled amount of data detail for increased scalability and hardware accelerator compatibility would be highly worthwhile.
