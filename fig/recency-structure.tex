\begin{figure*}
  \centering
    \includegraphics[width=\linewidth]{binder/binder-hstrat-reconstruction-quality/binder/teeplots/annotation-size=256+col=scenario+differentia-width=8+hue=kind+row=algo+scale=npop65536-ngen100000+viz=joyhist+x=time-ago+ext=}
\caption{%
\textbf{How does retention policy affect the structure of inner node loss?}
\footnotesize
Reconstruction node count densities for 256-bit size, byte differentia treatments.
Histograms depict internal node count relative to reference trees (dashed line), binned along the $x$ axis by time ago (ranging from most recent to most ancient, left-to-right).
Bars below the dashed line indicate that reconstructions produced \textit{fewer} inner nodes relative to ground truth (i.e., due to unresolved polytomies).
In contrast, bars reach above the dashed line where reconstruction produced \textit{more} internal nodes than ground truth.
Facet columns differentiate evolutionary regimes and facet rows correspond to steady, hybrid, and tilted retention policies.
Owing to its non-prioritization of retaining recent differentiae, steady policy loses nearly all resolution over recent (c. 100 generations) evolutionary history.
Tilted policy, on the other hand, reconstructs the highest proportion of recent history.
Hybrid policy does not suffer the catastrophic inner node loss over recent events seen under steady policy, but also reconstructs somewhat fewer recent inner nodes compared to pure tilted policy.
Adapted from \citep{moreno2025testing}.
}
  \label{fig:recency-structure}

\end{figure*}
